\documentclass[11pt]{article}
\usepackage[a4paper, total={6.8in, 9.25in}]{geometry}
\usepackage{graphicx}
\graphicspath{ {./assets/images/output/} }
\usepackage{caption}
\usepackage[english]{babel}
\usepackage{titling}
\usepackage{float}
\usepackage{amsmath}
\usepackage{minted}
\usepackage{multicol}
\usepackage{array}
\usepackage{setspace}
\usepackage{placeins}
\usepackage{tcolorbox}
\usepackage{caption}
\usepackage{subcaption}

\definecolor{verbgray}{gray}{.95}
\setminted{
    bgcolor=verbgray,
    baselinestretch=1,
    fontsize=\footnotesize,
    linenos,
    breaklines=true,
    breakanywhere=true,
    xleftmargin=\parindent
}
\title{Pixel-Level Manipulation and Histogram Analysis}
\author{}
\date{}

\pagenumbering{gobble}
\begin{document}
\captionsetup{justification=centering}
\input{dip_cover.tex}
\pagebreak
\tableofcontents
\pagebreak
\pagenumbering{arabic}
\maketitle

\section{Theory and Introduction}
In digital image processing, a fundamental operation is establishing relationships between pixels to define regions and boundaries. This relationship is governed by \textbf{connectivity}, which determines whether two pixels are considered "neighbors" \cite{gonzalez2018}. Finding the shortest path between two pixels is essential for tasks such as object labeling, region growing, and morphological processing.
This experiment investigates three types of connectivity to determine the shortest path between a Start point $S(x,y)$ and an End point $E(x,y)$ in a binary image:

\begin{itemize}
    \item \textbf{4-Connectivity ($N_4$):} Two pixels are connected if they share a horizontal or vertical edge. Coordinates: $(x \pm 1, y)$ and $(x, y \pm 1)$.
    \item \textbf{8-Connectivity ($N_8$):} Two pixels are connected if they share an edge or a corner (diagonal). Includes all neighbors in $N_4$ plus $(x \pm 1, y \pm 1)$.
    \item \textbf{m-Connectivity ($N_m$):} A modification of 8-connectivity designed to eliminate ambiguous multiple paths. Two pixels $p$ and $q$ are m-connected if:
          \begin{enumerate}
              \item $q$ is in $N_4(p)$, OR
              \item $q$ is in the diagonal neighborhood $N_D(p)$ \textbf{AND} their common 4-neighbors are empty (value 0) \cite{gonzalez2018}.
          \end{enumerate}
\end{itemize}

To find the optimal path, the Breadth-First Search (BFS) algorithm is utilized. Since the distance between adjacent pixels is uniform (1 step), BFS guarantees finding the shortest path by exploring the image layer-by-layer \cite{cormen2009}.

\section{Methodology}

\subsection{Problem Definition}
A $4 \times 4$ binary image matrix $I$ is analyzed, where $1$ represents the object and $0$ represents the background.
$$
    I = \begin{bmatrix}
        1 & 0 & 1 & 0 \\
        1 & 1 & 0 & 1 \\
        1 & 0 & 1 & 0 \\
        0 & 1 & 0 & 1
    \end{bmatrix}
$$
\textbf{Task:} The shortest path from coordinate $(1,1)$ to $(4,4)$ (using 1-based indexing) is to be found for all three connectivity types.

\subsection{Algorithm Implementation}
The Python implementation follows these steps:
\begin{enumerate}
    \item \textbf{Initialization:} The image matrix is defined and coordinates are converted to 0-based indexing ($Start=(0,0), End=(3,3)$).
    \item \textbf{Neighbor Selection:}
          \begin{itemize}
              \item For \textbf{4-Connectivity}, cardinal directions are checked only.
              \item For \textbf{8-Connectivity}, cardinal and diagonal directions are checked.
              \item For \textbf{m-Connectivity}, a specific condition is applied: diagonals are allowed only if the intersection of 4-neighbors is 0.
          \end{itemize}
    \item \textbf{Pathfinding (BFS):} A queue is initialized with the start node. The current path is iteratively dequeued, valid neighbors are checked, and new extended paths are enqueued until the destination is reached.
    \item \textbf{Visualization:} The resulting paths are plotted on the matrix grid using Matplotlib.
\end{enumerate}

\subsection{Python Code}
\begin{multicols}{2}
    \inputminted{python3}{./assets/lab3.py}
\end{multicols}


\section{Results}
The BFS algorithm was executed for each connectivity mode. The generated paths are visualized below:

\begin{figure}[H]
    \centering
    \includegraphics[width=\linewidth]{lab3_pathfinding.png}
    \caption{Shortest Path Analysis Results. Left: 4-Connectivity fails to find a path. Center: 8-Connectivity finds the shortest diagonal route. Right: m-Connectivity finds a valid but slightly longer path to resolve ambiguity.}
    \label{fig:results}
\end{figure}

\section{Discussion}
Distinct results were yielded for each connectivity type, highlighting their geometric properties:

\begin{itemize}
    \item \textbf{4-Connectivity (No Path):} A path failed to be found by the algorithm. As seen in the matrix, the Start pixel $(0,0)$ and its immediate neighbors form a cluster that is isolated from the destination $(3,3)$ when diagonals are forbidden. The zeros at $(1,2)$ and $(2,1)$ effectively act as a wall.
    \item \textbf{8-Connectivity (Length 4):} The shortest possible path was produced by this mode. By exploiting diagonal connections, the path moved directly from $(0,0) \to (1,1) \to (2,2) \to (3,3)$. This represents the Chebyshev distance.
    \item \textbf{m-Connectivity (Length 5):} The m-connectivity path was found to be longer than the 8-connectivity path.
          \begin{itemize}
              \item The move $(0,0) \to (1,1)$ was \textbf{invalid} in m-connectivity because the shared neighbor $(1,0)$ is $1$ (not empty). This forced the path to take a 4-connected step first: $(0,0) \to (1,0) \to (1,1)$.
              \item Subsequent moves $(1,1) \to (2,2)$ and $(2,2) \to (3,3)$ were valid diagonal m-connections because their shared neighbors were $0$.
          \end{itemize}
\end{itemize}

\section{Conclusion}
The impact of connectivity definitions on spatial analysis was successfully demonstrated in this lab. It was observed that 4-connectivity is the most restrictive, often failing to connect visually adjacent regions. 8-connectivity is the most permissive and yields the shortest geometric distance. m-connectivity serves as a robust hybrid, allowing diagonal connections only when necessary to prevent the ambiguity of multiple path options. Implementing these logic rules within a BFS framework effectively solved the shortest path problem.


\bibliographystyle{IEEEtran}
\renewcommand{\bibname}{References}
\addcontentsline{toc}{section}{References}
\bibliography{ref}

\end{document}